\documentclass[12,english]{article}
\usepackage[letterpaper, portrait, margin=1in]{geometry}

\usepackage{amsmath}
\usepackage[T1]{fontenc}
\usepackage{babel}
\usepackage{textcomp}
\usepackage{titlesec}
\setcounter{secnumdepth}{4}
\usepackage{hyperref}
\usepackage{xcolor}
\usepackage{booktabs}
\usepackage{placeins}
\usepackage{multirow}
\usepackage{tabularx}
\usepackage{booktabs}





\hypersetup{
    bookmarks=true,         % show bookmarks bar?
      colorlinks=true,       % false: boxed links; true: colored links
    linkcolor=black,          % color of internal links (change box color with linkbordercolor)
    citecolor=green,        % color of links to bibliography
    filecolor=magenta,      % color of file links
    urlcolor=cyan           % color of external links
}

\titleformat{\paragraph}
{\normalfont\normalsize\bfseries}{\theparagraph}{1em}{}
\titlespacing*{\paragraph}
{0pt}{3.25ex plus 1ex minus .2ex}{1.5ex plus .2ex}

\title{SE 3XA3: Module Interface Specification\\Jumbled Words}


\author{Team 08, Shunbill Jumble
	\\Shesan Balachandran, balacs1
	\\Bill Nguyen, nguyew3
	\\Muneeb Arshad, arsham14
}

\date{\today}

\begin{document}

\maketitle
\clearpage

\begin{table}[bp]
	\caption{\bf Revision History}
	\begin{tabularx}{\textwidth}{p{3cm}p{2cm}X}
		\toprule {\bf Date} & {\bf Version} & {\bf Notes}\\
		\midrule
		March 15 2021 & 0.0 & Initial Draft\\
		\bottomrule
	\end{tabularx}
\end{table}


\clearpage

\tableofcontents
\newpage

\section{Module Hierarchy}
\begin{table}[ht!]
\centering
\begin{tabular}{p{0.3\textwidth} p{0.6\textwidth}}
\toprule
\textbf{Level 1} & \textbf{Level 2}\\
\midrule

{Hardware-Hiding Module} & ~ \\
\midrule

\multirow{3}{0.3\textwidth}{Behaviour-Hiding Module}
& Main GUI Module \\
& Settings Module \\ % difficulty, modes, categories,
& Settings GUI Module \\ % showLeaderboard, showMainMenu, showGameModes, showDifficulty, showCategories
& Game Control Module \\ %guessing words, submit, change words, answer
& User GUI Module \\ %input username 


\midrule

\multirow{3}{0.3\textwidth}{Software Decision Module} 
& User Module\\ % update to leaderboard according to scores
& Leaderboard Module\\ % sort highest scores
\bottomrule

\end{tabular}
\caption{Module Hierarchy}
\label{TblMH}
\end{table}

\section{MIS of Main GUI Module}

\subsection{Uses}
Uses LeaderBoard Module
		\subsection{Interface Syntax}
		\subsubsection{Exported Access Programs}
		\begin{tabular}[pos]{|c|c|c|c|}
			
			\hline
			%	\label
			\textbf{Name}& \textbf{In} & \textbf{Out} & \textbf{Exceptions} \\ \hline
			init & - & - & -\\ \hline
			showLeaderboard & - & GUI Display & -\\ \hline
			quitGame & - & - & - \\ \hline
			start & - & - & -\\ \hline
			

		\end{tabular}
		
		\subsection{Interface Semantics}
		\subsubsection{State Variables}
		\subsubsection{Environmental Variables}
		\subsubsection{Assumptions}
    	
		
		\subsubsection{Access Program Semantics}
		init():
		
		Input: None
		
		Transition: Show main menu GUI
		
		Output: None
		
		Exceptions: None\\
		\\
        showLeaderboard():
		
		Input: None
		
		Transition: displays leaderboard screen when user wants to access the leaderboard via the main menu

		Output: Displays leaderboard screen
		
		Exceptions: None\\
		\\
		quitGame():
		
		Input: None
		
		Transition: Closes application
		
		Output: None
		
		Exceptions: None\\
		\\
		start():
		
		Input: None
		
		Transition: Show user GUI module
		
		Output: None
		
		Exceptions: None\\
		\\


\section{MIS of Settings GUI Module}
        \subsection{Uses}
        Uses Settings module
		\subsection{Interface Syntax}
		\subsubsection{Exported Access Programs}
		\begin{tabular}[pos]{|c|c|c|c|}
			
			\hline
			%	\label
			\textbf{Name}& \textbf{In} & \textbf{Out} & \textbf{Exceptions} \\ \hline
			init & - & GUI Display & -\\ \hline
			optionGameMode & - & GUI Display & - \\ \hline
			optionDifficulty & - & GUI Display & - \\ \hline
			optionCategory & - & GUI Display & - \\ \hline
			
		\end{tabular}
		
		\subsection{Interface Semantics}
		\subsubsection{State Variables}
		\subsubsection{Environmental Variables}
		\subsubsection{Assumptions}
		%It is assumed the user has accessed this GUI through the mainGUI menu. \\
		
		\subsubsection{Access Program Semantics}
		
		init():
		Input: None
		
		Output: Initialize GUI 
		
		Exceptions: None\\
		\\
	    optionGameMode():
		
		Input: None
		
		Transition: Uses settings module to select desired game mode (Ranked or Practice) and display game mode screen

		Output: Displays game mode screen and one of the two buttons that represent Ranked and Practice mode is selected to set desired game mode
		
		Exceptions: None\\
		\\
		optionDifficulty():
		
		Input: None
		
		Transition: Uses settings module to select desired difficulty level (easy, medium or hard) and display difficulty level screen

		Output: Display difficulty level screen and one of the three buttons that represent easy, medium and hard are selected to set the difficulty level of game 
		Exceptions: None\\
		\\
		optionCategories():
		
		Input: None
		
		Transition: Uses settings module to select desired category and display category screen

		Output: Display category screen and one of the categories button is selected and the game starts  
		Exceptions: None\\
		\\
\section{MIS of Settings Module}
\subsection{Uses}
		\subsection{Interface Syntax}
		\subsubsection{Exported Access Programs}
		\begin{tabular}[pos]{|c|c|c|c|}
			
			\hline
			%	\label
			\textbf{Name}& \textbf{In} & \textbf{Out} & \textbf{Exceptions} \\ \hline
			updateGameMode & int & - & - \\ \hline
			updateDifficulty & int & - & - \\ \hline
			updateCategories & int & - & - \\ \hline
			
		\end{tabular}
		
		\subsection{Interface Semantics}
		\subsubsection{State Variables}
		difficulty: int - Selected difficulty\\
		category: int - Selected category\\
		gamemode: int - Selected gamemode\\
		\subsubsection{Environmental Variables}
		\subsubsection{Assumptions}
		It is assumed the user has accessed this GUI through the mainGUI menu. \\
		
		\subsubsection{Access Program Semantics}

		updateGameMode(int gamemode):
		
		Input: integer value that is mappable to a gamemode
		
		Transition: Update the gamemode state variable

		Output: None
		Exceptions: None\\
		\\
		updateDifficulty(int difficulty):
		
		Input: integer value that is mappable to a difficulty
		
		Transition: Update the difficulty state variable

		Output: None
		Exceptions: None\\
		\\
		updateCategories(int category):
		
		Input: integer value that is mappable to a category
		
		Transition: Update the category state variable

		Output: None 
		Exceptions: None\\
		\\
		

		
\section{MIS of User GUI Module}
\subsection{Uses}
		\subsection{Interface Syntax}
		\subsubsection{Exported Access Programs}
		\begin{tabular}[pos]{|c|c|c|c|}
			
			\hline
			%	\label
			\textbf{Name}& \textbf{In} & \textbf{Out} & \textbf{Exceptions} \\ \hline
			init & - & GUI Display & -\\ \hline
			optionUserName & - & GUI Display & - \\ \hline
			submitUserName & str & GUI Display & - \\ \hline
			
			
		\end{tabular}
		
		\subsection{Interface Semantics}
		\subsubsection{State Variables}
		userName : str - username of the current player
		\subsubsection{Environmental Variables}
		\subsubsection{Assumptions}
		It is assumed the user has accessed this GUI through the mainGUI menu. \\
		
		\subsubsection{Access Program Semantics}
		
		init():
		
		Input:None
		
		Output: The GUI display of the username menu, showcasing the username configuration options
		
		Exceptions: None\\
		\\
		optionUsername():
		
		Input:None
		
		Output: Display textbox and submit button to allow players to enter and submit username
		
		Exceptions: None\\
		\\
		submitUsername(str username):
		
		Input: String input of the player's username
		
		Transition: Updates the state variable username with corresponding username
		
		Output: Uses the GUI module to show the category option
		
		Exceptions: None\\
		\\

	
\section{MIS of User Module}
        \subsection{Uses}
        userData: JSON file of user data consisting of username and scores
		\subsection{Interface Syntax}
		\subsubsection{Exported Access Programs}
		\begin{tabular}[pos]{|c|c|c|c|}
			
			\hline
			%	\label
			\textbf{Name}& \textbf{In} & \textbf{Out} & \textbf{Exceptions} \\ \hline
            addUser & String & - & FileNotFound\\ \hline
			updateScore & String, int & - & FileNotFound\\ \hline
			
		\end{tabular}
		
		\subsection{Interface Semantics}
		\subsubsection{State Variables}
	
		
		\subsubsection{Environmental Variables}

		\subsubsection{Assumptions}
		
		\subsubsection{Access Program Semantics}
		
		addUser(String username): 
		
    	Input: Username of the player
    	
    	Transition: Appends the username of the player and score value as null to the JSON userData file if it does not exists
    	
    	Exceptions: FileNotFound if the JSON file can not be found
    	
    	Output: -\\
		\\
		updateScore(String username, int score):
		
		Input: Username and score of the player

		Transition: Updates the new score of the player with corresponding username if the previous score is null or lower than the current score
		
		Output: - \\
		
\section{MIS of Leaderboard Module}
		\subsection{Interface Syntax}
		\subsubsection{Exported Access Programs}
		\begin{tabular}[pos]{|c|c|c|c|}
			
			\hline
			%	\label
			\textbf{Name}& \textbf{In} & \textbf{Out} & \textbf{Exceptions} \\ \hline
            getScores & file & array & FileNotFound\\ \hline
			getleaderboard & array & array & -\\ \hline
			
		\end{tabular}
		
		\subsection{Interface Semantics}
		\subsubsection{State Variables}
		
		topUsers: Array of tuples of size 2 - tuple[0] is the username and tuple[1] is the user score
		
		\subsubsection{Environmental Variables}

		\subsubsection{Assumptions}
		
		\subsubsection{Access Program Semantics}
		
		getScores(File userData): 
		
    	Input: Json file of user data with username and scores
    	
    	Transition: Reads JSON file of userData and store it in topUsers
    	
    	Exceptions: FileNotFound if the file name is invalid
    	
    	Output: topUsers\\
		\\
		getLeaderboard(topUsers):
		
		Input: Array of tuples topUsers that has user data

		Transition: sorts the topUsers from highest to lowest scores and outputs the first 10 tuples
		
		Output: topUsers\\
		\\
\section{MIS of Game Control Module}

\subsection{Uses}
Uses Settings Module
		\subsection{Interface Syntax}
		\subsubsection{Exported Access Programs}
		\begin{tabular}[pos]{|c|c|c|c|}
			
			\hline
			%	\label
			\textbf{Name}& \textbf{In} & \textbf{Out} & \textbf{Exceptions} \\ \hline
			init & - & - & -\\ \hline
			showCurrentQuestion & - & Display question & -\\ \hline
			getWordList & - & - & - \\ \hline
			getGuess & str & - & -\\ \hline
			isGuessCorrect & str & bool & -\\ \hline
			isTimeOut & - & bool & -\\ \hline
			clickedBack  & - & Display previous page & -\\ \hline
			clickedNext & - & Display next question  & -\\ \hline
			clickedCheckWord & - & Display result & -\\ \hline
			

		\end{tabular}
		
		\subsection{Interface Semantics}
		\subsubsection{State Variables}
	    guessedWord: str - Representation of user's current guess\\
		timeOut: int - Time left until the user can guess the word\\
		currentWord: str - Target value for the user's guess\\

		\subsubsection{Environmental Variables}
		
		\subsubsection{Assumptions}
    	Difficulty and category are selected before starting the game.\\
		
		\subsubsection{Access Program Semantics}
		init():
		
		Input: None
		
		Transition: Get values from the Settings module and sets state variables to their required values
		
		Output: None
		
		Exceptions: None\\
		\\
		showCurrentQuestion():
		
		Input: None
		
		Transition: Display question
		
		Output: None
		
		Exceptions: None\\
		\\
				\\
		getWordList():
		
		Input: None
		
		Transition: Extracts the suitable words for the current game based on settings
		
		Output: None
		
		Exceptions: None\\
		\\
		getGuess(str guess):
		
		Input: String input of the user's guess
		
		Transition: Updates the state variable for the current guess
		
		Output: None
		
		Exceptions: None\\
		\\
		isGuessCorrect(str guess):
		
		Input: String input of the user's guess
		
		Transition: Validates whether the users guess matches the target word
		
		Output: Boolean result of match or no match
		
		Exceptions: None\\
		\\
		isTimeOut():
		
		Input: None
		
		Transition: None
		
		Output: Checks if the user has timed-out for the current word
		
		Exceptions: None\\
		\\
		clickedBack():
		
		Input: None
		
		Transition: None
		
		Output: Display previous page
		
		Exceptions: None\\
		\\
		clickedNext():
		
		Input: None
		
		Transition: None
		
		Output: Display next question 
		
		Exceptions: None\\
		\\
		clickedCheckWord():
		
		Input: None
		
		Transition: None
		
		Output: Display result of guess 
		
		Exceptions: None\\
		\\
		



\end{document}