\documentclass[12pt, titlepage]{article}

\usepackage{booktabs}
\usepackage{tabularx}
\usepackage{hyperref}
\hypersetup{
    colorlinks,
    citecolor=black,
    filecolor=black,
    linkcolor=red,
    urlcolor=blue
}
\usepackage[round]{natbib}

\title{SE 3XA3: Test Plan\\Jumbled Words}


\author{Team 08, Shunbill Jumble
	\\Shesan Balachandran, balacs1
	\\Bill Nguyen, nguyew3
	\\Muneeb Arshad, arsham14
}

\date{\today}


\begin{document}

\maketitle

\pagenumbering{roman}
\tableofcontents
\listoftables
\listoffigures

\newpage
\begin{table}[bp]
	\caption{\bf Revision History}
	\begin{tabularx}{\textwidth}{p{3cm}p{2cm}X}
		\toprule {\bf Date} & {\bf Version} & {\bf Notes}\\
		\midrule
		March 04 2021 & 0.0 & Initial Draft\\
		\bottomrule
	\end{tabularx}
\end{table}

\clearpage

\pagenumbering{arabic}

\section{General Information}

\subsection{Purpose}

The purpose of this test plan is to outline and show how our Jumbled Words product will be tested. This document is important since proper/detailed testing of any software product is necessary in order to ensure that the product will function as intended according to the Software Requirements Specification (SRS). The previous document which was the SRS was completed prior to this document in order to form the basis for the creation of these tests. The testing plan document will be used during the creation of the product/software in order to validate and check that the system that is being created will satisfy all the necessary requirements. Also, this document will be used when writing the necessary software tests and when testing the product overall in order to ensure everything is satisfied.

\subsection{Scope}

The test plan will provide the outline for testing the functionality for the modification of the Jumbled Word Game. The scope of testing will cover the front-end components and the accuracy of their interactions and the accuracy of the backend logic which includes the game itself, user state management and the overall business logic of the product. These three categories will have details testing to cover specific requirements and product functionalities.

\subsection{Acronyms, Abbreviations, and Symbols}
	
\begin{table}[hbp]
\caption{\textbf{Table of Abbreviations}} \label{Table}

\begin{tabularx}{\textwidth}{p{3cm}X}
\toprule
\textbf{Abbreviation} & \textbf{Definition} \\
\midrule
SRS & Software Requirement Specification\\
POC & Proof of Concept\\
FR & Functional Requirement\\
NFR & Non-Functional Requirement\\

\bottomrule
\end{tabularx}

\end{table}


\subsection{Overview of Document}

The following sections of this document will be outlining and describing how our product “Jumbled Words” will be tested. Section 2 will consist of the main/overall test plan that will outline and describe what the specific software is, who will test the software, what are tools/software that will be used to perform these tests and what the test schedule will be. For section 3, this section will describe what functional and non-functional test cases that will be created for every area of the product/system and will also provide a traceability matrix for these test cases which will trace them to the system requirements. Section 4 will provide what was presented during the POC demonstration and how what was demonstrated was tested. Section 5, will be a comparison from the existing implementation to what is going to be added and improved on for this product. Section 6, will outline how the unit tests will be created and these unit tests will test the individual methods and functionality of the product. Finally, section 7 is the appendix of this document, which will just provide any additional information such as symbolic parameters and whether there is a usability survey. 


\section{Plan}
	
\subsection{Software Description}

The Jumbled Words Quiz In Python is a simple project for helping your kids grow in IQ. The project contains only the user side. The user can start the quiz by clicking on the start button. Also, you can choose the type of words, you want to solve in the quiz. The user can change jumbled words if they do not know the correct word. The players get points for guessing each word. The players can also access the leaderboard from the main menu to view player high scores. The game has different game modes and difficulty levels to challenge the player.

\subsection{Test Team}

The test team consists of all the members in Group 08, Shesan Balachandran, Bill Nguyen and Muneeb Arshad. 

\subsection{Automated Testing Approach}

The automated testing approach is to develop test suites with automated testing frameworks and apply regression testing techniques as new features are developed. The automated testing shall be managed through a script that would run all the test suites and output the testing results. Managing all the testing through a script will allow the team to test using a standardized approach throughout the team but also improve testing speeds as well. 

\subsection{Testing Tools}

We will be using the testing framework pytest to test the Python code. To test the statement coverage and the branch coverage we will be using coverage.py. Moreover to detect incorrect naming conventions of variables and functions, unused code/variables and to access the complexity of the code we will be using the vscode built-in static analysis tool for python called pylint.

\subsection{Testing Schedule}

All tasks that need to be accomplished are covered in the \href{https://gitlab.cas.mcmaster.ca/balacs1/se3xa3-project/-/tree/master/ProjectSchedule}{Gantt Chart} provided in the project planning, which includes the allocation of time for specific tasks as well as the deadlines.
		
\section{System Test Description}
	
\subsection{Tests for Functional Requirements}

\subsubsection{User Input}
		
\paragraph{Screen Interactions}
\begin{enumerate}

\item{FR-SCRN-1\\}

Type: Functional, Manual
					
Initial State: The start screen is displayed
					
Input: Click start game button
					
Output: Game screen switches from the start screen to options screen
					
How test will be performed: The tester selects the start game button on the main menu and the test asserts if the user interface switches and displays the option screen
					

\item{FR-SCRN-2\\}

Type: Functional, Manual

Initial State: Main screen

Input: Click Leaderboard Button 

Output: Leaderboard is loaded

How test will be performed: The tester selects the Leaderboard button on the main menu and checks if the user interface switches and displays the leaderboard screen

\item{FR-SCRN-3\\}

Type: Functional, Manual

Initial State: Main screen

Input: Quit Button

Output: The application shuts down 

How test will be performed: The tester selects the Quit button on the main menu and the checks if the game application shuts down

\item{FR-SCRN-4\\}

Type: Functional, Manual

Initial State: Screen with back button

Input: Click Back Button 

Output: The previous page is displayed

How test will be performed: To ensure all the back buttons have proper functionality we have to test the back button on each page. The tester will click the back button on the options screen and check the user interface switches to the Main Menu. Similarly, the back button will be clicked on the difficulty options screen and game mode options screen to check whether they go back to the previous page.
\end{enumerate}

\paragraph{Gamplay Interactions}
\begin{enumerate}	
	\item{FR-GMPY-1\\}
	
	Type: Functional, Manual
	
	Initial State: Empty input string
	
	Input:  Type the word using keyboard
	
	Output:  The typed word is displayed on screen
	
	How test will be performed: An expected text is typed through the keyboard and the actual typed word is asserted with the expected word
	
	
	\item{FR-GMPY-2\\}
	
	Type: Functional, Manual
	
	Initial State: Null state for the feedback notification
	
	Input: Click submit button 
	
	Output: The result of the submission (correct/incorrect) will be displayed in the notification.
	
	How test will be performed: For a sample question, the submit button is clicked and is checked with whether the state for the feedback notification is changed from null to showing state.
	
	\item{FR-SCRN-3\\}
	
	Type: Functional, Manual
	
	Initial State: A question is displayed on screen
	
	Input: Click next question button
	
	Output: If the correct answer was submitted already, change screens else remain on same screen
	
	How test will be performed: For a sample question, the submit button is clicked with a correct and incorrect answer, and is checked whether the screen changes for the answers respectively	
\end{enumerate}

\paragraph{Settings Menu}

\begin{enumerate}	
	\item{FR-STNG-1\\}
	
	Type: Functional, Manual
	
	Initial State: Difficulty Level Option Screen
	
	Input:  Click Easy Difficulty
	
	Output:  Timer corresponds to the difficulty level
	
	How test will be performed: The tester selects the Easy difficult button on the main menu then selects the random category and checks if the User Interface displays the game play screen and the timer corresponds to the difficulty level selected
	
	\item{FR-STNG-2\\}
	
Type: Functional,Manual

Initial State:Difficulty Level Option Screen

Input:Click Medium Difficulty

Output: Timer corresponds to the difficulty level

How the test will be performed: The tester selects the Medium difficult button on the main then selects the random category menu and checks if the User Interface displays the game play screen and the timer corresponds to the difficulty level selected. 	
	
	\item{FR-STNG-3\\}

Type: Functional, Manual.

Initial State:Difficulty Level Option Screen

Input:Click Hard Difficulty

Output: Timer corresponds to the difficulty level

How the test will be performed: The tester selects the Hard difficult button on the main menu then selects the random category and checks if the User Interface displays the game play screen and the timer corresponds to the difficulty level selected

\item{FR-STNG-4\\}

Type: Functional, Dynamic.

Initial State:Sample username

Input:Enter Username

Output: Username exists in the json file

How the test will be performed: A unit test will be performed by entering different inputs(string, integers, special characters) as username and check whether the username gets stored in the username json file

\item{FR-STNG-5\\}

Type: Functional, Manual

Initial State:Username screen

Input:Enter Username

Output: User Interface switches to next screen

How the test will be performed: The tester types in the username and checks whether the screen displays the typed username. The user selects the enter button and ensures the user interface switches to the game mode option screen. 

\item{FR-STNG-6\\}

Type: Functional, Manual

Initial State:Categories option screen

Input:Select Category

Output: User interface switches to game play

How the test will be performed: The tester is on the category options screen and can see all the available categories. Then a random category is selected and the user interface switches to another screen and displays the game play. The tester verifies the sample words matches the category selected. 

\item{FR-STNG-7\\}

Type: Functional, Manual

Initial State:Game Mode Option Screen

Input:Select Game Mode

Output: User interface switches display 

How the test will be performed: The tester is on the game mode options screen and can see all the available game modes. Then a specific game mode(practice, timed) is selected and the user interface switches to another screen and displays the difficult level option screen. The tester will proceed to the game play screen by selecting all other game settings and verify the game play has a timer when the timed mode is selected and no timer exists when practice mode is selected.

\end{enumerate}

\subsubsection{Game Logic}

\begin{enumerate}
	\item{FR-GMLG-1\\}
	
	Type: Functional, Static, Automatic
	
	Initial State: A sample question and a timer is selected
	
	Input: A correct answer is submitted at a specific time
	
	Output: Certain amounts of points are calculated for the answer given at the time
	
	How the test will be performed: Unit tests will be developed to check whether points are calculated accurately based on the time in the timer.
	
	\item{FR-GMLG-2\\}
	
	Type: Functional, Static, Automatic
	
	Initial State: A sample question is provided
	
	Input: A correct answer and an incorrect is provided
	
	Output: The result of the submission (correct/incorrect) will be displayed in the notification
	
	How the test will be performed: Unit tests will be developed to check whether the sample words do indeed match the expected words
	
	
	\item{FR-GMLG-3\\}
	
	Type: Functional, Static, Automatic
	
	Initial State: A sample question is provided
	
	Input: No input is provided and the time runs out
	
	Output: The next button will become active to go to the next question
	
	How the test will be performed: Unit tests will be developed to check whether the state of next button clickability is matched with timeouts
\end{enumerate}

\subsubsection{Leaderboard Testing}

\begin{enumerate}
	\item{FR-LEAD-1\\}
	
	Type: Functional, Dynamic, Manual
	
	Initial State: The leaderboard has no high scores
	
	Input: a new score is accomplished
	
	Output: The leaderboard will display only the new score on the leaderboard
	
	How the test will be performed: We will clear the high scores of all the users manually before starting the game and complete the game, which will generate a new score. Then we will make sure that the new score is the only score on the leaderboard and is at the top.
	
	\item{FR-LEAD-2\\}
	
	Type: Structural, Dynamic, Manual
	
	Initial State: The leaderboard is full (has 10 entries)
	
	Input: a new score is accomplished that is lower than all of the scores displayed in the leaderboard
	
	Output: The leaderboard will not be updated
	
	How the test will be performed: We will manually fill the leaderboard until it is full, then we will produce a score that is lower than all of the scores displayed in the leaderboard and ensure that the leaderboard did not change
	
	\item{FR-LEAD-3\\}
	
	Type: Functional, Dynamic, Manual
	
	Initial State: The leaderboard is full (has 10 entries)
	
	Input: a new score is accomplished that is higher than any of the scores displayed in the leaderboard
	
	Output: The leaderboard will be updated, removing the lowest score in the leaderboard and adding the new score in the corresponding order (from greatest to least)
	
	How the test will be performed: We will manually fill the leaderboard until it is full, then we will produce a score that is higher than any of the scores displayed in the leaderboard and ensure that the leaderboard removed the lowest score that is displayed in the leaderboard and added the new score. Also the leaderboard should display the scores from greatest to least
	
	\item{FR-LEAD-4\\}
	
	Type: Functional, Dynamic, Manual
	
	Initial State: The leaderboard has 3 entries
	
	Input: a unique new score is accomplished that is less than the current lowest score on the leaderboard
	
	Output: The leaderboard will be updated, and should have 4 entries, with newest score being placed last and should be displayed from greatest to least
	
	How the test will be performed: We will manually fill 3 entries in the leaderboard, then we will produce a new unique score that is higher than any of the 3 entries in the leaderboard. Then we will check if leaderboard is displayed from greatest to least
	
	\item{FR-LEAD-5\\}
	
	Type: Functional, Dynamic, Manual
	
	Initial State: The leaderboard has 3 entries
	
	Input: a unique new score is accomplished that is higher than any of the current scores on the leaderboard
	
	Output: The leaderboard will be updated, and should have 4 entries, with the leaderboard being displayed from greatest to least
	
	How the test will be performed: We will manually fill 3 entries in the leaderboard, then we will produce a new unique score that is lower than the 3 entries in the leaderboard. Then we will check if the new unique score is below the other 3 scores
\end{enumerate}

\subsection{Tests for Nonfunctional Requirements}

\subsubsection{Usability and Humanity Requirements}
		
\begin{enumerate}
\item{ NFR-UH-1\\}

Type: Structural, Dynamic, Manual

Initial State: Game is installed on a local device, and launched using the executable and goes to the Main Menu of the game

Input/Condition: User is expected to navigate through the Main Menu page without assistance

Output/Result: Majority of the users are able to access, start and view different parts of the game via the Main Menu page intuitively and without assistance

How test will be performed: We will have a selected group of users that have no background of how the product works will be given the product in the Main Menu state and will be asked to perform different tasks from the Main Menu page, such as starting the game, accessing the leaderboard etc. and if these users can do these tasks in a short amount of time then this test shall be considered successful

\item{ NFR-UH-2\\}

Type: Structural, Static, Manual

Initial State: The 3 members/developers of the product have a great understanding of the product and implementation

Input: The members will each think of 2-3 questions about usability to test each other, and gage every members opinion on the usability of the product

Output: The members will answer yes or no to these questions and if majority of the members answer yes (2 out of 3 members), this question will be deemed correct

How test will be performed: Each member will think of 2-3 questions related to the usability of the product and they will answer the questions themselves, and the other members will answer the questions and if majority of the questions, the members agree that this is satisfied then the test will be a success
\end{enumerate}

\subsubsection{Performance}

\begin{enumerate}
	\item{ NFR-PER-1\\}
	
Type: Non Functional, Manual

Initial State: The game will be made available with all active functionality

Input: Every clickable action is attempted several times and full game is played in different modes and difficulty level

Output:The time taken for the buttons to respond and the next page to load is recorded. If any of the response is less than 2seconds the tests are failed

How the test will be performed: The tester of the game will ensure the game is loaded on the computer. Then the tester will perform all clickable events on the menu options and game play with different game modes. The other tester will watch and record the response time for each event and record the time. If the response is greater than 2seconds then test will be marked as failed.
\end{enumerate}

\subsubsection{Operational And Environmental Requirements}

\begin{enumerate}	
	\item{ NFR-OE-1\\} 
	
Type: Non Functional, Manual

Initial State: The game must be available online to download

Input: The game is downloaded on different windows version and macOS

Output: The game is supported on different operating systems

How the test will be performed: The tester will manually download the game available online on different operating systems (Windows version 20H2 and macOS version 10.15.7) and verify the game is supported with all features

	\item{ NFR-OE-2\\} 
	
Type: Non Functional, Manual

Initial State: The game must be available online to download

Input: The game is downloaded on computer with 1Gb ram and 1Gb disk space

Output: The game is supported on computer with 1Gb ram and 1Gb disk space

How the test will be performed: The tester will manually download the game available online on a computer with 1Gb ram and 1Gb disk space and verify the game is supported with all features
\end{enumerate}

\subsection{Traceability Between Test Cases and Requirements}
Traceability for the functional requirements and the test cases\\

FR01: FR-STNG-4

FR02: FR-STNG-7

FR03: FR-STNG-1,FR-STNG-2,FR-STNG-3

FR04: FR-STNG-6

FR05: FR-GMLG-1

FR06: FR-GMLG-2

FR07: FR-GMLG-1

FR08: FR-SCRN-3

FR09: FR-STNG-5

FR10: FR-GMLG-3

FR11: FR-LEAD-1,FR-LEAD-2,FR-LEAD-3,FR-LEAD-4,FR-LEAD-5

FR12: FR-SCRN-1\\
\\
Traceability for the non-functional requirements and the test cases\\

NFR2: NFR-UH-1

NFR12: FR-SCRN-1

NFR3 : NFR-PER-1 

NFR4.1: NFR-OE-1 

NFR4.2: NFR-OE-2



\section{Tests for Proof of Concept}

\subsection{Best Score Testing}
		
\begin{enumerate}

\item{POC-1\\}

Type: Functional, Manual

Initial State: Game is started with settings of choice

Input: The game is played 

Output: The final score is the highest score

How the test will be performed: A sample playthrough is made where the answers are already given prehand so the final score will result in the highest score possible				
\end{enumerate}

\subsection{Quit}

\begin{enumerate}
	
	\item{POC-2\\}
	
Type: Functional, Manual

Initial State: Game is started with settings of choice

Input: The quit button is selected

Output: The ends and the state is idempotent

How the test will be performed: A sample play-through is made but the game is closed mid way to validate the states remain idempotent
	
\end{enumerate}

\section{Comparison to Existing Implementation}	

The existing implementation had no testing included in the project. Therefore a whole set of test cases were developed to validate the existing software as well as the new features. Extensive testing is done on the front-end components as that is what is primarily kept from the existing product. 
				
\section{Unit Testing Plan}
		
\subsection{Unit testing of internal functions}

The python testing framework pytest will be used for the unit testing of internal functions. For functions that directly return values, assert statements will be used to directly check the output. For functions that do not directly return values or for the testing of the UI elements, separate state variables will be used and modified by the interactions, which will then be checked through the assert statements. For example, to test if the back button works, a separate state variable can be used to validate whether the back button changed the state since it is difficult to test that directly.
		
\subsection{Unit testing of output files}	

There are no output files for our product. 	

\end{document}